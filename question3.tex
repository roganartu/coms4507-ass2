\chapter{Question 3 - Application Code Assessment}

\section{Preparation}
Not much preparation was required for this part of the competition, as the
members of the team were already quite confident in analysing and debugging
application code. Research was undertaken into common problems that can lead to
code being vulnerable to various attacks - for example, buffer overflows that
can allow an attacker to inject arbitary executable code into an application.

\section{Given Information}
Question 3 centred around an imaginary protocal called VSTP, for the Very 
Simple Transfer Protocol. The protocol was implemented in the Python 
programming language, and used the gnupg library for public key cryptography
to establish a secure channel to transfer a 256 bit sesion key, which was then
used to encrypt the rest of the session with AES in cipher feedback mode.

The program then presented a basic command line interface which allowed a user
to transfer files, remove files from the server, list directories, etc.

There was a server running VSTP: 172.16.1.20 port 5678
\\\\
All teams were given the file: vstp-package.zip
It contained the VSTP package, which comprised of the following:
\\- VSTP RFC
\\- Server and Client - Python source code
\\- The server public key
\\- Test client private / public keys (Key password is "password")
\\- Binary Debian package
\\- Installation script and README (please read before use)
\\\\
We were also given an unprivileged login to the server (VSTP):\\
\textbf{Username:} testclient
\\
\textbf{Password:} !321Password
\\\\
There was a server running Snort IDS: 172.16.1.107
\\\\
We were also given an unprivileged login to the server (SSH):\\
\textbf{Username:} test
\\
\textbf{Password:} Password123

section{Question 3.1}
\textbf{There is a command injection vulnerability in the VSTP server code
(vstp\_server.py). Review the code and exploit the vulnerability to retrieve the
flag from the ".garbage." file}
\subsection{Approach}
An analysis of the code quickly revealed that there was a vulnerability in the
move file (mv) command - when the server received the command from the user,
it used the system command to execute the system's move command 

\subsection{Retrospective Approach}
None?

\section{Question 3.1.1}
\textbf{INSERT MITIGATION QUESTION}
\subsection{Approach}
The mitigaion for this 

\section{Question 3.2}
\textbf{Located with the previous flag is a PCAP file containing a VSTP session.
Exploit the encryption vulnerability in the server code to decrypt the session,
extract the file transferred and retrieve the flag}
\subsection{Approach}
At first glance, the VSTP protocol seems quite secure, as it is using a 256
bit key, which is too large to brute force and will not be able to be cracked
until a crytographic weakness on AES is found. As AES is currently one of the
most commonly used ciphers today, it is out of the scope of the competition to
find such a weakness in it.

The vulnerability instead concerned the way the session key was generated. If
the key had been properly random, there would have been no way to crack it. But
the code that generated the key was very badly designed - the first 80 bits of
the key was hardcoded to always be the string "VSTPSERVER", then the next 64
were the date (in ASCII characters, formatted as YYYYMMDD), followed by the IP
address of the client in ASCII with the dot characters removed (32 to 96 bits,
depending on how long the IP was), then a two digit random number between 10
and 74, and the rest with actually random data (produced from the urandom 
function).

What this unusal key generation method produced was a key with very little
entropy. The client IP and date were able to be gathered from the network
capture, so straight up the first 216 bits were known. This meant that
only 40 bits needed to be brute forced, which was easily acheivable. The
fact that 16 of those bits were known to be an ASCII number between 10
and 40 reduces the entropy in the key by 10 bits to only 30 bits.

A known plaintext was required to be able to run the brute force, and this
was easily found in the network packet capture. Since the packets were not
padded at all, the message length was the length of the unencrypted message
plus the 16 bit initialisation vector. This meant it was very easy to tell
which messages were which because some were unique fixed sizes. The 'exit'
command was used, as it was very obvious which packet contained that message
given that it was the correct length and the server closed the connection
immediately after receiving it.

A script was written in Python to brute-force the ...

After the key had been found, the messages could be decrypted. There were
messages in the capture that fitted the size of a file transfer, so the
previous message had to be the command to initiate the transfer. This
assumption proved to be correct, and was decrypted to 'get flag.pdf'.

The subsequent packets (there were many of them) were copied out of the file
as hexadecimal strings, and reconstructed in a text editor. This text was then
taken into Python and decoded using the functions from the VSTP code and the
key found earlier. This produced a PDF file, which contained an image containing
the text of the flag - 'ArmsHungerGiftQuick149'.

\subsection{Retrospective Approach}
If more care had been taken in understanding the source code, some time could
have been saved - initially, it was thought that the server used its own IP
in the session key, but it actually used the client's address. Had this been
realised earlier, several runs of the brute force script could have been
prevented.

If there had been more time, it may have been more efficient to write a
script to take the PCAP file and reconstruct the data for the PDF file
instead of copying it out of Wireshark manually packet by packet. Although
the time taken writing such a script could easily have taken longer than
doing it manually.

\section{Question 3.2.1}
\textbf{INSERT MITIGATION QUESTION}
\subsection{Approach}
Approach details go here

\section{Question 3.3}
\textbf{There is a buffer overflow vulnerability in the Snort VSTP Preprocessor
code. Review the code and exploit the vulnerability to retrieve the flag from
the snort home directory}
\subsection{Approach}
It was very easy to find the buffer overflow in the C code that was provided.
It was the result of a strcpy call in a logging function. When a block that
looked like a GPG header was detected by the IDS, it would log the block,
temporarily copying it into a 799 byte buffer. If the message was malformed
and longer than this size, it would overflow the bounds of the buffer.

It was quite simple to exploit this at a simple level, to just crash the
process. This proved that it was definitely overflowing, and that arbirary
code would be able to be injected. None of the members on the team were
able to exploit the vulnerability though, as although research had been done
into how to detect and mitigate buffer overflows had been done, it was not
known how the shellcode injected into the application was designed.

\subsection{Retrospective Approach}
Had some more research and practice exploiting buffer overflows been done, the
team may have been able to formulate a payload that could be injected into
the applicaiton.

\section{Question 3.3.1}
\textbf{Unknown mitigation question}
\subsection{Approach}
Approach details go here
