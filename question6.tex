\chapter{Question 6 - VPN Security Analysis}

\section{Preparation}
The process of setting up a VPN on our own hardware before the competition was
conducted for fun, to give us a platform on which to practice and also to help
use develop a better understanding of VPN configurations and setups for this
section. We spent almost a week wading through a huge amount of configuration
documentation and guides before ultimately putting the VPN together. 

\section{Given Information}
All teams were given a file: vpn-solution.zip\\
It contained a network diagram and the client/server OpenVPN configuration.
\\\\
For each of these areas we had to identify how the current configuration
implements these functions, identify weaknesses in the current configuration and
detail changes that would improve the security of the deployed solution.

\section{Mitigation Questions 6.1 - 6.5}
\subsection{Retrospective Approach}
We left this section far too late in the competition to be able to successfully
complete all the questions, starting on Q6.1 with a little over 90 minutes left.
This is likely because the scoring system seemed to put a higher importance on
retrieving flags than answering mitigation questions, so we focused most of our
efforts on flags instead. Doing it over again we would likely have spent more
time on these questions at a far earlier stage, as they were relatively easy to
answer, just time consuming to do so.

\section{Mitigation Question 6.1}
\textbf{Review how the current configuration implements authentication, identify
any weaknesses and detail changes that would improve the security of the
deployed solution}
\subsection{Approach}
As there is such a huge range of OpenVPN configuration settings, there was no
real way to just look at the files provided and properly understand what they
were doing. In order to get a better overview of what each individual setting
did we resorted to the OpenVPN man pages \cite{OpenVPNMan}, searching for each
setting and writing down what it did. As the man pages are very verbose,
containing example values for each setting, as well as outlining potential
security issues when using certain values, it was relatively straightforward to
determine where errors could possibly be found.

We then drew up a map of the network, with the settings at each and what effect
that would have had on the connection. By doing this we were able to
successfully determine the issues with authentication was the use of static
pre-shared keys instead of a handshake-based protocol such as TLS. We advised
that they switch to an authentication protocol that utilises a certificate
authority, such as TLS.

\section{Mitigation Question 6.2}
\textbf{Review how the current configuration implements encryption, identify any
weaknesses and detail changes that would improve the security of the deployed
solution}
\subsection{Approach}
Using an approach similar to above we were able to successfully determine that
the issue with encryption was the use of an insecure, outdated cipher (namely
DES-CBC). We suggested that they utilise a crytographically secure encryption
method instead, such as the default OpenVPN cipher, AES.

\section{Mitigation Question 6.3}
\textbf{Review how the current configuration implements VPN termination points,
identify any weaknesses and detail changes that would improve the security of
the deployed solution}
\subsection{Approach}
Due to time constraints, we were unable to begin this question.

\section{Mitigation Question 6.4}
\textbf{Review how the current configuration implements routing, identify any
weaknesses and detail changes that would improve the security of the deployed
solution}
\subsection{Approach}
Due to time constraints, we were unable to begin this question.

\section{Mitigation Question 6.5}
\textbf{Review how the current configuration implements logging, identify any
weaknesses and detail changes that would improve the security of the deployed
solution}
\subsection{Approach}
Using an approach similar to above we were able to successfully determine that
the issue with logging was that the serveside log file was saved in a user
directory. User directories in UNIX are by default, world readable, and hence
the log file would be able to be read by any user logged into the system. We
advised that they move the log file into a system folder or one accessible only
by root, as well as changing the permissions to ensure that only privileged
users have any access to the log files.
