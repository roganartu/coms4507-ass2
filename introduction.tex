\chapter{Introduction}

The Cyber Security Challenge Australia (CySCA) is an information security
competition run annually as a joint effort by the Defence Signals Directorate
(DSD) and Telstra. The competition was first run in 2012 with 15 teams fighting
for the title of CySCA champion, as well as flights and entry to one of the
world's leading information security conferences, Black Hat.

2013 drew almost triple the number of entrants, with 43 teams competing for
their chance to fly to Vegas.  The competition ran for 24 hours from 12 noon
Tuesday 7th May to 12 noon Wednesday 8th May  (AEST) and comprised of six main
sections; Web Application Pentest, Corporate Network Pentest, Application Code
Assessment, Memory Forensics, Network Forensics and Vulnerability Assessment.

Within each section was a number of questions, each with their own weighting.
The questions were mostly flags to 'capture' from the various systems, with one
section dedicated entirely to answering questions about a given VPN
configuration file.  Most of the flags, when submitted, unlocked a further
question about how we would suggest mitigating the issue or vulnerability that
allowed us to gain access and capture the flag.

The competition was designed to be run completely autonomously, similar to a
standard information security Capture The Flag (CTF) game. Each team was given
access to the competition environment via their own VPN connection, giving each
group their own environment to work and test within. In order to ensure a fair
game the scoring server as well as other team's environments were defined as
being out of scope and hence were off limits to pen testing efforts.

The scenario for CySCA 2013 was to perform penetration testing for a company
called Synergised Cyber Cloud Pty Ltd, an information security company looking
to ensure their website, corporate network and application source all meet their
high security standards. In order to prepare, teams were advised to practice
using a number of free tools including BackTrack, IDA, gdb, objdump, wireshark,
Volatility and Metasploit, as well as to read up on cryptanalysis and the APT
lifecycle.

This paper is a retrospective exploration of our entire CySCA 2013 experience,
from the long pre-competition preparation to the longer post-competition sleep.
